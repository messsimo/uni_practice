\documentclass[a4paper,12pt]{report}

\usepackage[romanian]{config}

% Description
\newcommand{\thesisTitleEng}{Numele tezei} % Doar pentru master
\newcommand{\uniGroupName}{MIA2201}

\newcommand{\authorName}{Nume Prenume} % Fără patronimic
\newcommand{\thesisTitle}{Explorarea formatului PNG}
\newcommand{\thesisType}{master} % an / licență / master. <teză de> se adaugă automat.
\newcommand{\programulDeStudii}{master} % licență / master. <programul de> se adăugă automat.
\newcommand{\identificatorulCursului}{0613.5 Informatică aplicată} % 0211.7 Design jocurilor
\newcommand{\conducatorName}{Nume Prenume}

\renewcommand{\year}{2024}
\newcommand{\conferencesList}{Conferința Studențească, Editia NUMARUL-a, \year}
\newcommand{\github}{\url{https://github.com/USER/REPO}}
\newcommand{\outputDate}{\today}

\begin{document}

\titlePage

\clearpage
\tableofcontents

\clearpage
\unnumberedChapter{Adnotare} % Doar pentru master

\textbf{la \thesisType{} ``\thesisTitle{}'', a studentului \authorName{}, grupa \uniGroupName{}, programul de studii \programulDeStudii.}

\textbf{Structura tezei.}
Teza constă din: Introducere, \chapterCount{} capitole, Concluzii generale și recomandări, Bibliografie \bibliographyEntryCount{} titluri.
Textul de bază cuprinde \usefulPageCount{} de pagini și \anexeCount{} anexe.

\textbf{Cuvinte-cheie:}
% \acs înseamnă o referință la o prescurtare cu este.
% Face ca cuvântul să devină click-abil în PDF-ul generat.
\textit{\acs{PNG}, cuvânt 1, cuvânt 2}

\textbf{Actualitatea.}

Unde se folosesc tehnologiile, de ce e importantă temă.

\textbf{Scopul și obiectivele cercetării.}

Scopul = ce încercați să demonstrați prin această teză.

Obiectivele = lucrurile prin care demonstrați asta.

\textbf{Rezultatele preconizate și obținute} rezumă în: 
(1) studierea ceva
(2) proiectarea ceva
(3) implementarea ceva.

\textbf{Problemele importante rezolvate} sunt: ...

\textbf{Valoarea aplicativă.} Ca rezultat s-a obținut ...

Toate codurile sursă a proiectului pot fi accesate pe GitHub după următorul link: \github.

Rezultatele obținute au fost raportate la \textbf{\conferencesList}\cite{self}.

\clearpage
\unnumberedChapter{Annotation} % Doar pentru master

\textbf{of the \thesisType{} thesis  ``\thesisTitleEng'' of the student \authorName{}, group \uniGroupName{}, \programulDeStudii{} study program.}

\textbf{The structure of the thesis.}
The thesis is made up of: Introduction, \chapterCount{} chapters,
General conclusions and recommendations, Bibliography of \bibliographyEntryCount{} titles.
Base text takes up \usefulPageCount{} pages and has \anexeCount{} appendicies.

\textbf{Keywords:}
\textit{}

\textbf{Relevance.}

\textbf{Purpose and objectives of this research.}

\textbf{The planned and obtained results}

\textbf{Important resolved problems} are:

\textbf{Applicative value.} 

The entire source code can be obtained by accessing the GitHub repository
by the following URL: \github. 

The obtained results were reported at \textbf{\conferencesList}\cite{self}.

\clearpage
\unnumberedChapter{Lista Abrevierilor}
\begin{acronym}[JPEG]
    % Adăugați prescurtările aici.
    % Adăugați doar prescurtările menționate.
    % În viitor, mereu scrieți \ac{PNG} în loc de pur și simplu PNG.
    \acro{PNG}{Portable Network Graphics}
    \acro{PR}{Pull Request}
\end{acronym}

\introChapter

\textbf{Actualitatea și importanța temei.}

Tot aceeași chestie ca și la adnotare, dar mai pe lung.

\textbf{Scopul și obiectivele.}

Tot aceeași.

\textbf{Suportul metodologic și tehnologic.}

Ce fel de librării / instrumente / resurse ați folosit.

Ce fel de abordare ați luat.

Poate inspirațiile.


\textbf{Noutatea stiințifică/originalitatea.}

Ce vă diferă teza / produs de alte aplicații / soluții / studii deja existente.

\textbf{Valoarea aplicativă.}

Pentru ce poate fi folosită aplicația.
Faptul că v-a ajutat să studiați aceste tehnologii nu ajunge, 
trebuie să mai specificați ceva în afară de asta.

Puteți face și astfel de listă:
\begin{itemize}
    \item Valoarea A
    \item Valoarea B
    \item Valoarea C
\end{itemize}

\textbf{Sumarul tezei.}

Primul capitol, \nameref{intro_chapter_title}, prezintă informații generale / teoretice despre ...

Al doilea capitol, \nameref{architecture_chapter_title}, schițează implementarea, ...

Al treilea capitol, \nameref{implementation_chapter_title}, urmează implementarea ...

\chapter{Denumirea capitolului teoretic}\label{intro_chapter_title}

\section{Sintaxa \LaTeX{}}

\subsection{Introducere}

Aici se demonstrează sintaxa \LaTeX{} de bază.

Evident că subcapitole de așa mărime nu sunt acceptabile, 
ar trebui să argumentați lucrurile mai mult.

De exemplu, s-ar putea scrie că sintaxa de bază este importantă
pentru a putea scrie chiar o lucrare primitivă în \LaTeX{}, și deoarece
scopul este să cunoască \LaTeX{} la nivel destul de avansat pentru a putea
scrie teza, este aceste informații sunt esențiale pentru a putea proceda la
niște idei mai adânce.

\subsection{Citarea surselor}

Citarea întregei surse se face după bucata de text care o menționează, așa\cite{gif_unusable_reason}.

Puteți face referință la pagini concrete, sau la paragrafe, așa\cite[12.2.]{png_spec}.

\subsection{Sintaxa listelor}

Neordonate. Folosiți pentru prezentarea unor informații legate:

\begin{itemize}
    \item
        Imagini cu paletă de culori de până la 256 de culori.
        
    \item
        Posibilitatea de a fi transmis în flux:
        fișierele pot fi citite și scrise în serie, permițând astfel utilizarea formatului
        PNG ca protocol de comunicare pentru generarea și afișarea imaginilor dinamic.

    \item
        Afișare progresivă: un fișier de imagine pregătit corespunzător poate fi
        afișat pe măsură ce este primit pe canalul de comunicare,
        furnizând foarte repede o imagine cu rezoluție redusă,
        urmată de îmbunătățirea graduală a detaliilor.

    \item
        Transparență: porțiuni ale imaginii pot fi marcate ca transparente,
        creând efectul unei imagini non-rectangulare.

    \item 
        Informații auxiliare: comentarii textuale și alte date pot fi
        stocate în interiorul fișierului de imagine.

    \item
        Complet independenți de hardware și platformă.

    \item
        Compresie eficientă, 100\% fără pierderi.
\end{itemize}

Ordonate. Folosiți de exemplu pentru o listă de pași consecutivi:

\begin{enumerate}
    \item 
        Imagini cu paletă de culori de până la 256 de culori.
    \item 
        Posibilitatea de a fi transmis în flux:
        fișierele pot fi citite și scrise în serie, permițând astfel utilizarea formatului
        PNG ca protocol de comunicare pentru generarea și afișarea imaginilor dinamic.
    \item 
        Afișare progresivă: un fișier de imagine pregătit corespunzător poate fi
        afișat pe măsură ce este primit pe canalul de comunicare,
        furnizând foarte repede o imagine cu rezoluție redusă,
        urmată de îmbunătățirea graduală a detaliilor.
    \item 
        Transparență: porțiuni ale imaginii pot fi marcate ca transparente,
        creând efectul unei imagini non-rectangulare.
\end{enumerate}


\subsection{Stilizarea}

\textbf{TEXT Bold Face = Bold}

\textit{TEXT ITalic = Italic}

\texttt{TEXT TeleType = font Monospace pentru cod}

Puteți să le \textbf{aplicați} \textit{într-o propoziție}, sau să le \texttt{\textit{\textbf{combinați}}}.


\subsection{Tabele}

Când aveți un tabel, trebuie să dați referința la el.
Exemplu\refFigure{pixel_order_table} de un tabel: 

\begin{figure}[!ht]
\centering
\begin{tabular}{c c c c c c c c}
    1 & 6 & 4 & 6 & 2 & 6 & 4 & 6 \\
    7 & 7 & 7 & 7 & 7 & 7 & 7 & 7 \\
    5 & 6 & 5 & 6 & 5 & 6 & 5 & 6 \\
    7 & 7 & 7 & 7 & 7 & 7 & 7 & 7 \\
    3 & 6 & 4 & 6 & 3 & 6 & 4 & 6 \\
    7 & 7 & 7 & 7 & 7 & 7 & 7 & 7 \\
    5 & 6 & 5 & 6 & 5 & 6 & 5 & 6 \\
    7 & 7 & 7 & 7 & 7 & 7 & 7 & 7 \\
\end{tabular}
\caption{Tabelul ordonării pixelilor}
\label{fig:pixel_order_table}
\end{figure}


\section{Librăria \texttt{minted} pentru cod}

\subsection{Informații generale}

Librăria \texttt{minted} se folosește pentru a stiliza codul.

\subsection{Inserarea directă a unui bloc de cod}

\begin{minted}{cpp}
// orice cod aici
int main()
{
    std::cout << "hello";
}
\end{minted}

\subsection{Inserarea unui fișier întreg}

\inputminted[]{zig}{../src/sourcefile.zig}

\subsection{Inserarea unei bucăți din fișier}

\inputminted[firstline=2,lastline=5]{zig}{../src/sourcefile.zig}

\subsection{Inserarea unui segment din fișier}

Această funcționalitate este realizată datorită script-ului \texttt{findSegment.py}.
\texttt{minted} poate adăuga așa funcționalitate în viitor (sau deja este?).

Pentru a configura limbajul, modificați \texttt{findSegment.py}.

\inputMintedSegment{../src/sourcefile.zig}{example}

\section{Imagini}

\subsection{Introducere}

Acest șablon propune doar o funcție pentru imagini, care ține și de figure automat.
Puteți folosi altceva dacă nu vă aranjează.
Pentru standardizare, însă, ar fi bine să vă propuneți metoda cu un \ac{PR} ca s-o adăugați în șablon.

\subsection{O imagine centrată}

Când dați o imagine, la ea numaidecât să faceți și referința.

În \refFigure{interface_sketch.png} este prezentată o schiță a interfeței grafice.

\imageWithCaption{interface_sketch.png}{Schița interfeței grafice}

\section{Anexe}

\subsection{Când se folosește o anexă vs codul direct?}

Atunci când conținutul codului nu este strict esențial 
pentru a-l explica la nivel înalt în text.

Încă atunci, când codul este prea mare. 
Încercați să includeți direct în text cam 50 de linii maxim.

\subsection{Cum se face referința}

A se vedea anexa \ref{appendix:example_min}.

\chapterConclusionSection{intro_chapter_title}

Ce s-a discutat în acest capitol, pe scurt.


\chapter{Proiectarea aplicației}\label{architecture_chapter_title}

\section{Introducere}

Aici includeți informații despre structura aplicației la nivel înalt, 
poate niște diagrame, de ce ați folosit tehnologiile pe care le-ați folosit, etc.

\chapterConclusionSection{architecture_chapter_title}

\chapter{Implementarea Sistemului}\label{implementation_chapter_title}

\section{Introducere}

Aici includeți mai multe detalii despre module concrete din sistemul implementat.
Ce probleme ați întâlnit și cum le-ați depășit.

Cum ați folosit funcționalitățile importante din librării.

Prezentați aplicația cu imagini, ori direct în text, 
dacă ele sunt importante pentru înțelegere și mai specifice,
ori cu referințe la anexe.

\chapterConclusionSection{implementation_chapter_title}

În acest capitol s-au discutat ...

\unnumberedChapter{Concluzii Finale și Recomandări}

Ce s-a discutat pe scurt, reamintiți cele mai interesante momente.
Mai adăugați ce ar putea fi îmbunătățit sau unde poate fi folosit produsul în viitor.

Un exemplu al unei teze complet:
\begin{itemize}
  \item Sursa (programul, textul tezei în \LaTeX{}): \url{https://github.com/AntonC9018/thesis-png}
  \item PDF-ul: \url{https://drive.google.com/file/d/1ZiGQt6PvUm3FGY8oyIlVsERc4PvZHJu0/view?usp=drive_link}
\end{itemize}

Acestea 2 de mai jos le includeți mereu:

Toate codurile sursă, inclusiv codurile de program și textul lucrării
în forma înainte de randare, pot fi accesate pe GitHub după următorul link: \github.

Rezultatele obținute au fost raportate la \textbf{\conferencesList}\cite{self}.

\bibliographyChapter

\appendixChapter

\section{Funcția min într-un program test JavaScript}\label{appendix:example_min}
\inputminted{js}{../src/appendix_example.js}

\declarationPage{}

\end{document}
% vim: fdm=syntax
